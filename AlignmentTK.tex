\documentclass[12pt]{article}
%\usepackage{amsfonts}
%\usepackage{amssymb}
%\usepackage{amsmath}
%\usepackage{listings}
%\lstset{language=[LaTeX]TeX}

\begin{document}
\title {The Alignment Toolkit}
\author{Ara V. Nefian and Taemin Kim\\
\{ara.nefian,taemin.kim\}@nasa.gov}
\maketitle

This document describes a basic use structure of the AlignmentTK (ATK). ATK is developed for 2D and 3D data alignment and supports the 
following applications:
\begin{itemize}
\item assembler - aligns two DEMs, to be used in rover localization with a large orbital map. It uses iterative clsoest point (ICP) algorithm and has 
been tested for MER imagery in the context of HiRISE orbital maps.
\item lidar2dem - aligns DEM to lidar data. It uses ICP and has been tested on several DEMs and LOLA tracks.
\item lidar2img - aligns lidar to images. It uses a affine transform to match the LOLA simulated reflectance to Apollo imagery. 
\end{itemize} 

\section{How to use}
Examples of use are given in assembler.sh, lidar2dem.sh and lidar2img.sh respectively.

Each application is controlled by an optional settings file (see coregister\underline{ }settings\underline{ }example.txt) with the following example format:\\
MATCHING\underline{ }MODE 1\\
REFLECTANCE\underline{ }TYPE 2\\
ANALYSE\underline{ }FLAG 0\\
USE\underline{ }REFLECTANCE\underline{ }FEATURES 1\\
TOP\underline{ }PERCENT\underline{ }FEATURES 1\\
SAMPLING\underline{ }STEP 100 100\\
MATCH\underline{ }WINDOW 5 5\\
MAX\underline{ }NUM\underline{ }ITER 15\\
MAX\underline{ }NUM\underline{ }STARTS 20\\
NO\underline{ }DATA\underline{ }VAL -10000.0\\
CONV\underline{ }THRESH 0.01\\

If the settings file is not found or not entered by the user each application will run with a set of default params.

Each field of the settings file is explained below:
\begin{itemize}
\item{MATCHING\underline{ }MODE}: 0 - no matching, 1 - affine 2D alignmment, 2 -ICP 3D alignment\\
\item{ANALYSE\underline{ }FLAG}:  1 - will display and save verbose information, 0 - no info.\\
\item{USE\underline{ }REFLECTANCE\underline{ }FEATURES}: 0 - no LOLA features, 1 use LOLA features.\\ 
\item{TOP\underline{ }PERCENT\underline{ }FEATURES} integer value representing 1000 x perecntage of top features to be kept from LOLA data\\
\item{SAMPLING\underline{ }STEP} two integers corresponding to the downsampling factor in the width and height of the image or DEM\\
\item{MATCH\underline{ }WINDOW}: two integers corresponding to the width and height of the matching window\\
\item{MAX\underline{ }NUM\underline{ }ITER} 15\\
\item{MAX\underline{ }NUM\underline{ }STARTS} 20\\
\item{NO\underline{ }DATA\underline{ }VAL} no data float value for images or DEMs that don't have geotiff specified no data value.\\
\item{CONV\underline{ }THRESH}: float number describing the threshold under which absolute value at consecutive iterations determine 
                                the algorithm convergence (example: 0.01)\\
\end{itemize}


\section{Code Structure}
The following files are part of the toolkit
\begin{itemize}
\item{lidar2img.cc} - main function for lidar2img
\item{lidar2dem.cc} - main function for lidar2dem 
\item{assembler.cc} - main function for assembler
\end{itemize}



\section{SFM Processing}
The following files are part of the SFM framework:
\begin{itemize}
	\item{sfm\_test.cpp} - main function for SFM Processing
	\item{SFM.h} - declaration of SFM class
	\item{SFM.cpp} - definition of SFM class
	\item{PoseEstimation.h} - declaration of PoseEstimation class
	\item{PoseEstimation.cpp} - definition of PoseEstimation class
	\item{FeatureExtraction.h} - declaration of FeatureExtraction class
	\item{FeatureExtraction.cpp} - definition of FeatureExtraction class
	\item{sfm\_config.txt} - configuration file for main program
\end{itemize}


\noindent How to install SFM:
\begin{enumerate}
	\item{Install Prerequisites} - Install OpenCV 2.3.1, PCL 1.5
	\item{Install SFM} - inside sfm processing directory type: cmake .
	\item{Build sfm\_test} - inside sfm processing directory type: make
	\item{Run Examples} - ./sfm\_test testConfig.txt testImageList.txt results
\end{enumerate}

\end{document}
%=======
%\section{Installation of SFM}

%This is complete installation procedure for Mac OS X.

%\begin{lstlisting}
%- install prerequisites
%1) install Macport (http://www.macports.org/install.php) 
%2) install dependencies (http://pointclouds.org/downloads/macosx.html)
%3) install PCL (http://pointclouds.org/downloads/macosx.html)
%4) sudo port install opencv
%5) sudo port install cmake
%6) install gfortran (http://gcc.gnu.org/wiki/GFortranBinaries)
%7) install lapack(http://gcc.gnu.org/testing/testing-lapack.html)
%7.1) download lapack.tgz and unzip it
%7.2) rename make.inc.example to make.inc in the root directory of lapack 
%7.3) make blaslib && make
%8) install sba (http://www.ics.forth.gr/~lourakis/sba/) 
%8.1) edit demo/CMakeLists.txt

%# CMake file for sba's demo program

%INCLUDE_DIRECTORIES(..)
%LINK_DIRECTORIES(.. ${LAPACKBLAS_DIR})

%ADD_EXECUTABLE(eucsbademo eucsbademo.c imgproj.c readparams.c eucsbademo.h readparams.h)
%# libraries the demo depends on
%IF(HAVE_F2C)
%  TARGET_LINK_LIBRARIES(eucsbademo sba ${LAPACK_LIB} ${BLAS_LIB} ${F2C_LIB})
%ELSE(HAVE_F2C)
%  TARGET_LINK_LIBRARIES(eucsbademo sba ${LAPACK_LIB})
%ENDIF(HAVE_F2C)

%# make sure that the library is built before the demo
%ADD_DEPENDENCIES(eucsbademo sba)

%8.2) mkdir build && cd build
%8.3) ccmake ..
%8.3.1) turn HAVE_F2C OFF
%8.3.2) LAPACK_LIB = -framework vecLib
%8.4) make
%8.5) set PATH for the libsba.a

%For beginner, check opt/local if you install them with "sudo port install." Otherwise, usr/local

%- Initialize SFM
%1) svn co https://babelfish.arc.nasa.gov/svn/stereopipeline/sandbox/sfm/
%2) cd sfm
%3) mkdir build
%4) cd build
%5) cmake .. (here is big double dots)
%6) make -j2 && make install

%- Configure SFM
%1) cd sfm\build
%2) ccmake ..
%[Press 'c' to configure]
%[Select option 'BUILD_TESTS']
%[Press 'c' to reconfigure]
%[Press 'g' to generate]

%# Test SFM
%1) cd sfm/build
%2) make test
%3) view the file in sfm/build/Testing/Tempory/LastTest.log

%# Run SFM with an Example
%1) cd sfm/examples
%2) scp <username>@<machinename>:/irg/data/kinect/<directory> .
%3) cd <directory>
%4) cp ../sfm_config_example.txt sfm_config.txt
%5) cp ../camera_calibration_example.txt camera_calibration.txt
%6) open sfm_config.txt to change the value of inputDataName by <directory>
%7) sfm
%8) pc_vis <directory>.txt

%for example <directory>=2011_1202_bldg_269_2, <machinename>=pesto
%1) cd sfm/examples
%2) scp <username>@pesto:/irg/data/kinect/2011_1202_bldg_269_2 .
%3) cd 2011_1202_bldg_269_2
%4) cp ../sfm_config_example.txt sfm_config.txt
%5) cp ../camera_calibration_example.txt camera_calibration.txt
%6) open sfm_config.txt: inputDataName 2011_1202_bldg_269_2
%7) sfm
%8) pc_vis 2011_1202_bldg_269_2.txt

%# troubleshooting
%If any problem to link dependencies,
%Download PCL sources from http://pointclouds.org/downloads/ 
%- compile and install PCL
%1) cd PCL-<ver>-Source
%2) cmake .
%3) make
%4) make install

%Add the following to ~/.bash_profile file
%export DYLD_FALLBACK_LIBRARY_PATH=/usr/local/lib:/usr/lib
%\end{lstlisting}

%\end{document}
