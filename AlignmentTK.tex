\documentclass[12pt]{article}
%\usepackage{amsfonts}
%\usepackage{amssymb}
%\usepackage{amsmath}

\begin{document}
\title {The Alignment Toolkit}
\author{Ara V. Nefian\\
ara.nefian@nasa.gov}
\maketitle

This document describes a basic use structure of the AlignmentTK (ATK). ATK is developed for 2D and 3D data alignment and supports the 
following applications:
\begin{itemize}
\item assembler - aligns two DEMs, to be used in rover localization with a large orbital map. It uses iterative clsoest point (ICP) algorithm and has 
been tested for MER imagery in the context of HiRISE orbital maps.
\item lidar2dem - aligns DEM to lidar data. It uses ICP and has been tested on several DEMs and LOLA tracks.
\item lidar2img - aligns lidar to images. It uses a affine transform to match the LOLA simulated reflectance to Apollo imagery. 
\end{itemize} 

\section{How to use}
Examples of use are given in assembler.sh, lidar2dem.sh and lidar2img.sh respectively.

Each application is controlled by an optional settings file (see coregister_settings_example.txt) with the following example format:\\
MATCHING\underline{ }MODE 1\\
REFLECTANCE\underline{ }TYPE 2\\
ANALYSE\underline{ }FLAG 0\\
USE\underline{ }LOLA\underline{ }FEATURES 1\\
TOP\underline{ }PERCENT\underline{ }FEATURES 1\\
SAMPLING\underline{ }STEP 100 100\\
MATCH\underline{ }WINDOW 5 5\\
MAX\underline{ }NUM\underline{ }ITER 15\\
MAX\underline{ }NUM\underline{ }STARTS 20\\
DISPLAY\underline{ }RESULTS 1\\
NO\underline{ }DATA\underline{ }VAL -10000.0\\
CONV\underline{ }THRESH 0.01\\

If the settings file is not found or not entered by the user each application will run with a set of default params.

Each field of the settings file is explained below:
\begin{itemize}
\item{MATCHING\underline{ }MODE}: 0 - no matching, 1 - affine 2D alignmment, 2 -ICP 3D alignment\\
\item{ANALYSE\underline{ }FLAG}:  1 - will display and save verbose information, 0 - no info.\\
\item{USE\underline{ }LOLA\underline{ }FEATURES}: 0 - no LOLA features, 1 use LOLA features.\\ 
\item{TOP\underline{ }PERCENT\underline{ }FEATURES} integer value representing 1000 x perecntage of top features to be kept from LOLA data\\
\item{SAMPLING\underline{ }STEP} two integers corresponding to the downsampling factor in the width and height of the image or DEM\\
\item{MATCH\underline{ }WINDOW}: two integers corresponding to the width and height of the matching window\\
\item{MAX\underline{ }NUM\underline{ }ITER} 15\\
\item{MAX\underline{ }NUM\underline{ }STARTS} 20\\
\item{DISPLAY\underline{ }RESULTS}: 0 - no results display, 1 - results display\\
\item{NO\underline{ }DATA\underline{ }VAL} no data float value for images or DEMs that don't have geotiff specified no data value.\\
\item{CONV\underline{ }THRESH}: float number describing the threshold under which absolute value at consecutive iterations determine 
                                the algorithm convergence (example: 0.01)\\
\end{itemize}
\end{document}

\section{Code Structure}