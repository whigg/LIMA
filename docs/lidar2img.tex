The {\bf{lidar2img}} application coregisters LIDAR and image data to form a consistent map.
Specifically, it takes a set of Lunar Orbiter Laser Altimeter (LOLA) readings from
the Lunar Reconaissance Orbiter (LRO) and an Apollo metric camera image as input. It then
aligns the two data sources to form a consistent map, and outputs a transformation from
the original imagine coordinates to the adjusted image coordinates.

\subsection{Files:}

\begin{itemize}
	\item{lidar2img.cc, The main method, parses command line arguments and calls high level functions.}
	\item{display.cc / display.h, Routines to visualize the data and the alignment process.}
	\item{featuresLOLA.cc / featuresLOLA.h, Find key points to use as ground control points.}
	\item{match.cc / match.h, Gauss-Newton algorithm to align tracks to images. Also implements currently
		unused brute force search, and computes homographies between pairs of images, or two sets of LOLA
		track coordinates.}
	\item{tracks.cc / tracks.h, Define, load and save track data structures. Compute track reflectance and
		luminence, ann transform tracks.}
	\item{util.cc / util.h, Helpful utility functions.}
\end{itemize}

\subsection{How to install:}

\begin{enumerate}
	\item{\emph{Install Prerequisites}} The {\texttt{lidar2img}} application depends on VisionWorkbench and the
		Ames Stereo Pipeline, which in turn depends on ISIS. See the VisionWorkbench manual
		for how to compile and install VisionWorkbench. For ASP and ISIS, we recommend using
		the BaseSystem tarball. % TODO: is this an official thing even? explain this better
		For ISIS, you must install the Base data and the Apollo15 mission data. See
		\href{http://isis.astrogeology.usgs.gov/}{the ISIS web site} for details.
	\item{\emph{Install }}
		In order to compile {\bf{lidar2img}} you must set three environment variables:
		{\texttt{ISISROOT}} (which should already have been set during ISIS installation),
		{\texttt{VWROOT}}, and {\texttt{ASPROOT}}. The variables {\texttt{VWROOT}} and
		{\texttt{ASPROOT}} should point to the directories where VisionWorkbench and the
		Ames Stereo Pipeline have been installed, respectively. If the desired ISISROOT
		is not a subdirectory in the BaseSystem, you can set the {\texttt{BASESYSTEMROOT}}
		environment variable to choose a different BaseSystem location.
	\item{\emph{Build }}
		In the \texttt{lidar2img\_processing} directory, run ``\texttt{cmake .}'' followed by
		``\texttt{make}'' to create the {\texttt{lidar2img}} executable.
	\item{\emph{Basic Usage}}
		The command ``\texttt{lidar2img -l lola\_tracks.csv -i image.cub --outputImage alignment.png}'' will
		align the LOLA tracks in the CSV file \texttt{lola\_tracks.csv} to the image file \texttt{image.cub}.
		The transformation matrix will be output, and the resulting alignment will be shown in \texttt{alignment.png}.
\end{enumerate}

\subsection{Runtime Options}

The following options may be passed to {\texttt{lidar2img}} at runtime.

\begin{itemize}
	\item{\texttt{-l, --lidarFile FILE}} : a CSV file containing the LOLA shots to process
	\item{\texttt{-t, --tracksList FILE}} : a file containing a list of LOLA CSV shots to process, separated by linebreaks
	\item{\texttt{-i, --inputCubFile FILE}} : a cub image to align tracks to
	\item{\texttt{-o, --outputFile FILE}} : a file to output the transformation matrix to (if none is specified, stdout is used)
	\item{\texttt{--outputImage FILE}} : a file to output an image visualizing the aligned tracks to
	\item{\texttt{-d, --dataFile FILE}} : a file to output track data too, in format used for visualization script
	\item{\texttt{-g, --gcpDir DIRECTORY}} : a directory to output ground control points to
	\item{\texttt{-h, --help}} : display a help message
\end{itemize}

\subsection{The LIDAR to Image Alignment Algorithm}


