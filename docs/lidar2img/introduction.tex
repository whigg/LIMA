
Many different spacecraft have succesfully returned various types of extraplanetary data to Earth,
including imagery and elevation data. However, due to small uncertainties in space craft
position, these different data sources have errors in alignment which makes forming a
consistent multi-source map a challenging problem. Finding transformations from one data
set to another to form a consistent map is the {\emph{coregistration}} problem.

The problem of coregistering different sources of imagery data has been succesfully adressed
with bundle adjustment. However, the problem of coregistering different classes
of data, in particular aligning LIDAR measurements with images, has not been well-studied.
This coregistration problem is particularly challenging because the types of data are
fundamentally different. Furthermore, the available LIDAR data is much sparser than
the image data.

The {\texttt{lidar2img}} program coregisters images
from the Apollo 15 metric camera to data from the recently deployed Lunar Reconaissance Orbiter's
Lunar Orbiter Laser Altimeter (LOLA). To do so, we first convert the LOLA data to a
{\emph{synthetic image}} by inputting measured surface normals, and estimated sun and spacecraft
positions into a lunar reflectance model. Then, we find a planar homography which aligns the synthetic
image to the actual image using the Gauss-Newton algorithm.  However, the Gauss-Newton algorithm is
susceptible to local minima, and a naive application will fail. Instead, we first
apply Gauss-Newton to lower-resolution images which smooth over local minima. Then, we
refine the transformation on successively higher-resolution layers of the image pyramid.

We first present and discuss the data sources in detail. Next, we introduce the
Gauss-Newton algorithm as applied to the LIDAR to image coregistration problem,
and its use in conjunction with the image pyramid. Finally, we present selected
results demonstrating the effectiveness of the coregistration process.

