The {\bf dem2dem} application matches to DEMs by finding the best rotation matrix and translation vector. This method uses an Iterative Closest Point (ICP) solution.
The {\bf assembler} application aligns specifically a foreground DEM as obtained from rover traverse over an orbital DEM and overlays them. If orthoprojected image data is available, 
assembler will use the DEM alignment to overlap overlay the two (ground level and orbital) orthoprojected images. The assembled output is reelased as a set of DEM and DRG tiles. 

\subsection{Files:}
\begin{itemize}
\item{assembler.cc, main file for assembler application}
\item{assembler.h, header file for assembler application}
\item{icp.cpp, source file for icp functions}
\item{icp.h, header file for icp functions}
\end{itemize}

\subsection{How to install:}
\begin{enumerate}
	\item{Install Prerequisites} 
	\item{Install} 
	\item{Build} 
\end{enumerate}

\subsection{How to run}
./assembler config.txt
\subsubsection{Configuration Parameters}
If the configuration file is not specified, the program will run with a set of default parameters.
\begin{itemize}


	\item{\textsc{MATCHING\_MODE}}: 0 - no alignment, 1 - altitude alignment, 2-ICP with translation only, 3-ICP with rotation and translation; {\bf default}
        \item{\textsc{WEIGHTING\_MODE}}: 0 - no DEM weighting, 1 - DEM weights based on ground level DEM/ orbital DEM resolution; {\bf default}
        \item{\textsc{MAX\_NUM\_STARTS}}: the number of restarts for the alignment. this number must be an integer $n^2$ where $n$ is the number of restarts in vertical or horizontal direction; {\bf default}
        \item{\textsc{DELTA\_LON\_LAT}}: the lon lat dispacement for each restart of the alignment; {\bf default}
        \item{\textsc{USE\_LON\_LAT\_RAD\_OFFSET}}: 0 - ignore the initial offset of the foreground DEM in the line below, 1 - read the original DEM offset from the line below; {\bf default} 
        \item{\textsc{LON\_LAT\_RAD\_OFFSET}}: initial lon lat and radial offset of the foreground DEM; {\bf default}
        \item{\textsc{MAX\_FORE\_PPD}}: max foreground DEM pixel per degree resolution; {\bf default}
        \item{\textsc{SAMPLING\_STEP}}: sampling step in horizontal and vertical direction taken for matching features step in ICP; {\bf default}
	\item{\textsc{MATCH\_WINDOW}}:p ixel size in horizontal and vertical direction of the matching window used in ICP; {\bf default}
        \item{\textsc{MAX\_NUM\_ITER}}: max number of iterations used in ICP; {\bf default}
        \item{\textsc{CONV\_THRESH}}: convergence threshold used in ICP. When the ICP error at consecutive iterations falls below this threshold or 
                                      MAX\_NUM\_ITER is achieved the ICP iteration stop; {\bf default}
        \item{\textsc{TILE\_SIZE\_DEM}}: DEM tile size in horizontal and vertical direction compatible with the viewer; {\bf default} (Antares for MSL)
        \item{\textsc{PADDING\_PARAMS\_DEM}}: DEM tile padding top, left bottom, right compatible with the viewer; {\bf default} (Antares for MSL)
        \item{\textsc{TILE\_SIZE\_DEM}}: DRG tile size in horizontal and vertical direction compatible with the viewer; {\bf default} (Antares for MSL)
	\item{\textsc{PADDING\_PARAMS\_DRG}}: DRG tile padding top, left bottom, right compatible with the viewer; {\bf default} (Antares for MSL)
        \item{\textsc{FORE\_NO\_DATA\_VAL\_DEM}}: no data value for the foreground DEM; {\bf default}
        \item{\textsc{BACK\_NO\_DATA\_VAL\_DEM}}: no data value for the background DEM; {\bf default}
        \item{\textsc{FORE\_NO\_DATA\_VAL\_DRG}}: no data value for the foreground DRG; {\bf default}
        \item{\textsc{BACK\_NO\_DATA\_VAL\_DRG}}: no data value for the background DRG; {\bf default}

\end{itemize}


\subsection{The algorithm explained: Iterative Closest Point (ICP)}
The goal of ICP is to find a rotation matrix $R$ and a translation vector $T$ such that 
\begin{eqnarray}
{\bf r} = R*{\bf m}+T
\end{eqnarray}
where $\bf r$ is a reference and $\bf m$ is matching array of 3D vectors respectively.
At each iteration
\begin{itemize}
\item {\bf step 1:} determine best matches between features in $\bf r$ and $\bf m$.
\item {\bf step 2:} estimate $R$ from arrays ${\bf r} - E({\bf r})$ and ${\bf m}-E({\bf m})$ using Kaubsh method; $E({\bf x})$ is the expected value of $\bf x$;
                   note that these two arrays have 0 mean.
\item {\bf step 3:} estimate $T=E({\bf r}-R*{\bf m})$.
\item {\bf step 4:} estimate a new array $\tilde{\bf m} = R*{\bf m} + T$.
\item {\bf step 5:} check convergence, otherwise go to step 1.
\end{itemize}

%\begin{comments}
%An alternative solution:
%At each iteration
%\begin{itemize}
%\item {\bf step 1:} determine best matches between features in $\bf r$ and $\bf m$.
%\item {\bf step 2:} estimate $R$ from arrays ${\bf r}$ and ${\bf m}+E({\bf r}-{\bf m})$,  using Kaubsh method;
%                   note that these two arrays have the same mean; $E({\bf x})$ is the expected value of $\bf x$.
%\item {\bf step 3:} estimate $T=R*E({\bf r}-{\bf m})$.
%\item {\bf step 4:} estimate a new array $\tilde{\bf m} = R*{\bf m} + T$.
%\item {\bf step 5:} check convergence, otherwise go to step 1.
%\end{itemize}
%\end{comments}