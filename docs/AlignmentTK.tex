\documentclass[12pt]{article}
%\usepackage{amsfonts}
%\usepackage{amssymb}
%\usepackage{amsmath}
%\usepackage{listings}
%\lstset{language=[LaTeX]TeX}

\begin{document}
\title {The Alignment Toolkit}
\author{Ara V. Nefian \\ Taemin Kim\\
\{ara.nefian,taemin.kim\}@nasa.gov}
\maketitle

This document describes a basic use structure of the AlignmentTK (ATK). ATK is developed for 2D and 3D data alignment and supports the following applications:
\begin{itemize}
\item assembler - aligns two DEMs, to be used in rover localization with a large orbital map. It uses iterative clsoest point (ICP) algorithm and has 
been tested for MER imagery in the context of HiRISE orbital maps.
\item lidar2dem - aligns DEM to lidar data. It uses ICP and has been tested on several DEMs and LOLA tracks.
\item lidar2img - aligns lidar to images. It uses a affine transform to match the LOLA simulated reflectance to Apollo imagery.
\item stereo\_processing - creates 3D point clouds from stereo image pairs
\item sfm\_processing - performs structure from motion on pairs of images
\end{itemize} 

\section{How to install the Alignment Toolkit}
You can simply follow the first subsection to build and test the ATK once you installed its prerequisites such as OpenCV 2.3.1, PCL 1.5, lapack, cmake and gfortran. The complete installation procedure of the prerequisites on the Mac OS X is described in \ref{sec:How to install Prerequisites}.

\subsection{How to build and test the ATK:}
\begin{enumerate}
	\item{svn co  https://babelfish.arc.nasa.gov/svn/stereopipeline/sandbox/lima} - download the ATK
	\item{mkdir lima/build}
	\item{cd lima/build}
	\item{cmake ..} - here is big double dots 
	\item{make install} - executables are installed in usr/local/bin by default (see the section \ref{sec:How to change BIN directory} to change the location of executables.)
	\item{make test} - view the details in the file lima/build/Testing/Temporary/LastTest.log
\end{enumerate}

\subsection{How to run SFM:}
\begin{enumerate}
	\item{mkdir lima/examples}
	\item{cd lima/examples}
	\item{Copy images and their depth data with their file lists (e.g., testImageList.txt and testPCList.txt) and configuration file (e.g., testConfig.txt)}
	\item{sfm\_test testConfig.txt testImageList.txt results} - where testImageList.txt contains the list of input images
	* make sure the bin directory is in \$PATH.
\end{enumerate}

\subsection{How to install Prerequisites:}\label{sec:How to install Prerequisites}

\begin{enumerate}
	\item{} install Macport (http://www.macports.org/install.php) 
	\item{} install dependencies (http://pointclouds.org/downloads/macosx.html)
	\item{} install PCL (http://pointclouds.org/downloads/macosx.html)
	\item{} sudo port install opencv
	\item{} sudo port install cmake
	\item{} install gfortran (http://gcc.gnu.org/wiki/GFortranBinaries)
	\item{} install lapack(http://gcc.gnu.org/testing/testing-lapack.html)
	\begin{enumerate}
		\item{} download lapack.tgz and unzip it
		\item{} rename make.inc.example to make.inc in the root directory of lapack 
		\item{} make blaslib
		\item{} make
	\end{enumerate}
	\item{} install sba (http://www.ics.forth.gr/~lourakis/sba/) 
	\begin{enumerate}
		\item{edit demo/CMakeLists.txt as follows} -
		
			\# CMake file for sba's demo program

			INCLUDE\_DIRECTORIES(..)
			LINK\_DIRECTORIES(.. \$\{LAPACKBLAS\_DIR\})

			ADD\_EXECUTABLE(eucsbademo eucsbademo.c imgproj.c readparams.c eucsbademo.h readparams.h)
			
			\# libraries the demo depends on
			
			IF(HAVE\_F2C)
			
				TARGET\_LINK\_LIBRARIES(eucsbademo sba \\\$\{LAPACK\_LIB\} \$\{BLAS\_LIB\} \$\{F2C\_LIB\})
				
			ELSE(HAVE\_F2C)
			
				TARGET\_LINK\_LIBRARIES(eucsbademo sba \$\{LAPACK\_LIB\})
				
			ENDIF(HAVE\_F2C)

			\# make sure that the library is built before the demo

			ADD\_DEPENDENCIES(eucsbademo sba)

		\item{mkdir build}
		\item{cd build}
		\item{ccmake ..}
		\begin{enumerate}
			\item{turn HAVE\_F2C OFF}
			\item{LAPACK\_LIB = -framework vecLib}
		\end{enumerate}
		\item{make}
		\item{set PATH for the libsba.a}
	\end{enumerate}
\end{enumerate}

\subsection{How to change BIN directory}\label{sec:How to change BIN directory}

\begin{enumerate}
	\item{ccmake .} - in the ATK root directory (e.g., lima)
	\item{change CMAKE\_INSTALL\_PREFIX} - then executable files will be generated in CMAKE\_INSTALL\_PREFIX/bin directory.
\end{enumerate}

\section{How to use}
Examples of use are given in assembler.sh, lidar2dem.sh and lidar2img.sh respectively.

Each application is controlled by an optional settings file (see coregister\_settings\_example.txt) with the following example format:\\
MATCHING\_MODE 1\\
REFLECTANCE\_TYPE 2\\
ANALYSE\_FLAG 0\\
USE\_REFLECTANCE\_FEATURES 1\\
TOP\_PERCENT\_FEATURES 1\\
SAMPLING\_STEP 100 100\\
MATCH\_WINDOW 5 5\\
MAX\_NUM\_ITER 15\\
MAX\_NUM\_STARTS 20\\
NO\_DATA\_VAL -10000.0\\
CONV\_THRESH 0.01\\

If the settings file is not found or not entered by the user each application will run with a set of default params.

Each field of the settings file is explained below:
\begin{itemize}
\item{MATCHING\_MODE}: 0 - no matching, 1 - affine 2D alignmment, 2 -ICP 3D alignment\\
\item{ANALYSE\_FLAG}:  1 - will display and save verbose information, 0 - no info.\\
\item{USE\_REFLECTANCE\_FEATURES}: 0 - no LOLA features, 1 use LOLA features.\\ 
\item{TOP\_PERCENT\_FEATURES} integer value representing 1000 x perecntage of top features to be kept from LOLA data\\
\item{SAMPLING\_STEP} two integers corresponding to the downsampling factor in the width and height of the image or DEM\\
\item{MATCH\_WINDOW}: two integers corresponding to the width and height of the matching window\\
\item{MAX\_NUM\_ITER} 15\\
\item{MAX\_NUM\_STARTS} 20\\
\item{NO\_DATA\_VAL} no data float value for images or DEMs that don't have geotiff specified no data value.\\
\item{CONV\_THRESH}: float number describing the threshold under which absolute value at consecutive iterations determine 
                                the algorithm convergence (example: 0.01)\\
\end{itemize}


%\section{Code Structure}
%The following files are part of the toolkit
%\begin{itemize}
%\item{lidar2img.cc} - main function for lidar2img
%\item{lidar2dem.cc} - main function for lidar2dem 
%\item{assembler.cc} - main function for assembler
%\end{itemize}


\section{SFM Processing}
\subsection{Files:}
\begin{itemize}
	\item{sfm\_test.cpp} - main function for SFM Processing
	\item{SFM.h} - declaration of SFM class
	\item{SFM.cpp} - definition of SFM class
	\item{PoseEstimation.h} - declaration of PoseEstimation class
	\item{PoseEstimation.cpp} - definition of PoseEstimation class
	\item{FeatureExtraction.h} - declaration of FeatureExtraction class
	\item{FeatureExtraction.cpp} - definition of FeatureExtraction class
	\item{sfm\_config.txt} - configuration file for main program
\end{itemize}


\subsection{How to install sfm:}
\begin{enumerate}
	\item{Install Prerequisites} - Install OpenCV 2.3.1, PCL 1.5, lapack, cmake, gfortran
	\item{Install SFM} - inside sfm processing directory type: cmake .
	\item{Build sfm\_test} - inside sfm processing directory type: make
	\item{Run Examples} - ./sfm\_test testConfig.txt testImageList.txt results
\end{enumerate}

\subsection{Configuration Parameters}
If the configuration file is not specified, the program will run with a set of default parameters.
\begin{itemize}
	\item{\textsc{depthInfo}} - 0 (No Depth), 2 (Kinect Depth), 3 (Stereo Point Clouds)
	\item{\textsc{pointCloudFilename}} - file containing filenames for all point cloud files
	\item{\textsc{kinectDepthFilename}} - file containing list of kinect depth filenames
	\item{\textsc{minMatches}} - minimum number of matches in order to do pose estimation
	\item{\textsc{detThresh}} - small threshold for determinant of R in order to do SVD
	\item{\textsc{featureMethod}} - 0 (SURF), 1 (ORB), 2 (SIFT)
	\item{\textsc{poseEstimationType}} - 0 (Pairs of frames for Pose Estimation), 1 (SBA)
	\item{\textsc{nndrRatio}} - ratio for determining good matches
	\item{\textsc{numNN}} - number of nearest neighbors to save for each key point
	\item{\textsc{zDist}} - maximum distance between z coordinates for good matches
	\item{\textsc{xDist}} - maximum distance between x coordinates for good matches
	\item{\textsc{yDist}} - maximum distance between y coordinates for good matches
	\item{\textsc{nbMatches}} - minimum number of matches for homography
	\item{\textsc{homographyMethod}} - 0 (Default OpenCV Method), 4 (Least Median Method), 8 (RANSAC Method)
	\item{\textsc{flannCheck}} - number of elements to visit in NN search
	\item{\textsc{ransacPixels}} - maximum number of pixels the match can be from the epipolar line
	\item{\textsc{ransacAccuracy}} - accuracy of homography before stopping iterations
	\item{\textsc{tileWidth}} - width of each tile in image
	\item{\textsc{tileHeight}} - height of each tile in image
	\item{\textsc{xOverlap}} - amount of overlap in the x direction between consecutive tiles
	\item{\textsc{yOverlap}} - amount of overlap in the y direction between consecutive tiles
	\item{\textsc{tileScaleX}} - scale from reference tile to matching tile in the x direction
	\item{\textsc{tileScaleY}} - scale from reference tile to matching tile in the y direction
	\item{\textsc{pointProjFilename}} - file where point projections are saved for later use by SBA
	\item{\textsc{weightedPortion}} - weights this portion of the image to the bottom of the image (work in progress \ldots)
\end{itemize}

\end{document}


%=======
%\section{Installation of SFM}

%This is complete installation procedure for Mac OS X.

%\begin{lstlisting}
%- install prerequisites
%1) install Macport (http://www.macports.org/install.php) 
%2) install dependencies (http://pointclouds.org/downloads/macosx.html)
%3) install PCL (http://pointclouds.org/downloads/macosx.html)
%4) sudo port install opencv
%5) sudo port install cmake
%6) install gfortran (http://gcc.gnu.org/wiki/GFortranBinaries)
%7) install lapack(http://gcc.gnu.org/testing/testing-lapack.html)
%7.1) download lapack.tgz and unzip it
%7.2) rename make.inc.example to make.inc in the root directory of lapack 
%7.3) make blaslib && make
%8) install sba (http://www.ics.forth.gr/~lourakis/sba/) 
%8.1) edit demo/CMakeLists.txt

%# CMake file for sba's demo program

%INCLUDE_DIRECTORIES(..)
%LINK_DIRECTORIES(.. ${LAPACKBLAS_DIR})

%ADD_EXECUTABLE(eucsbademo eucsbademo.c imgproj.c readparams.c eucsbademo.h readparams.h)
%# libraries the demo depends on
%IF(HAVE_F2C)
%  TARGET_LINK_LIBRARIES(eucsbademo sba ${LAPACK_LIB} ${BLAS_LIB} ${F2C_LIB})
%ELSE(HAVE_F2C)
%  TARGET_LINK_LIBRARIES(eucsbademo sba ${LAPACK_LIB})
%ENDIF(HAVE_F2C)

%# make sure that the library is built before the demo
%ADD_DEPENDENCIES(eucsbademo sba)

%8.2) mkdir build && cd build
%8.3) ccmake ..
%8.3.1) turn HAVE_F2C OFF
%8.3.2) LAPACK_LIB = -framework vecLib
%8.4) make
%8.5) set PATH for the libsba.a

%For beginner, check opt/local if you install them with "sudo port install." Otherwise, usr/local

%- Initialize SFM
%1) svn co https://babelfish.arc.nasa.gov/svn/stereopipeline/sandbox/sfm/
%2) cd sfm
%3) mkdir build
%4) cd build
%5) cmake .. (here is big double dots)
%6) make -j2 && make install

%- Configure SFM
%1) cd sfm\build
%2) ccmake ..
%[Press 'c' to configure]
%[Select option 'BUILD_TESTS']
%[Press 'c' to reconfigure]
%[Press 'g' to generate]

%# Test SFM
%1) cd sfm/build
%2) make test
%3) view the file in sfm/build/Testing/Tempory/LastTest.log

%# Run SFM with an Example
%1) cd sfm/examples
%2) scp <username>@<machinename>:/irg/data/kinect/<directory> .
%3) cd <directory>
%4) cp ../sfm_config_example.txt sfm_config.txt
%5) cp ../camera_calibration_example.txt camera_calibration.txt
%6) open sfm_config.txt to change the value of inputDataName by <directory>
%7) sfm
%8) pc_vis <directory>.txt

%for example <directory>=2011_1202_bldg_269_2, <machinename>=pesto
%1) cd sfm/examples
%2) scp <username>@pesto:/irg/data/kinect/2011_1202_bldg_269_2 .
%3) cd 2011_1202_bldg_269_2
%4) cp ../sfm_config_example.txt sfm_config.txt
%5) cp ../camera_calibration_example.txt camera_calibration.txt
%6) open sfm_config.txt: inputDataName 2011_1202_bldg_269_2
%7) sfm
%8) pc_vis 2011_1202_bldg_269_2.txt

%# troubleshooting
%If any problem to link dependencies,
%Download PCL sources from http://pointclouds.org/downloads/ 
%- compile and install PCL
%1) cd PCL-<ver>-Source
%2) cmake .
%3) make
%4) make install

%Add the following to ~/.bash_profile file
%export DYLD_FALLBACK_LIBRARY_PATH=/usr/local/lib:/usr/lib
%\end{lstlisting}

%\end{document}
